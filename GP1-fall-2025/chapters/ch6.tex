








\iffalse
\section{Conclusion}
In this section we are going to conclude the analysis and our study. We will remind you what was in this report.
\begin{itemize}
	\item Firstly, we introduced the study with the automobile's history including overview on the car as one of the complicated machines, then the history of transformation from the horses to the cars in means of reliability and comfortability in the road transportation. Then we talked about the vehicle suspension system components and its evolution and improvement. Also, the debit between the ride quality and vehicle handling was demonstrated.
	\item To get into studying and improving the vehicle suspension system, we investigated in the previous literatures talking about the motivation and purposes to push the automotive industry and researchers to enhance the passive suspension system.
	\item Then we provided the experimental work. In one hand, the mechanical part including the preparation of the real physical model, the experiments for passive suspension parameters such as control arm, wheel and tire assembly, and setup frame moment inertia. Then we got the wheel reaction and sprung mass. In the other hand, the electronic part including the sensors used with their specification and wiring to reveal the data.
	\item The mathematical model of passive suspension system section was talking about the three model dynamic models:
	\begin{itemize}
		\item Two DOF quarter-car models.
		\item Four DOF half-car models.
		\item Seven DOF full- car models
	\end{itemize}
	
	These are commonly used models for the theoretical analysis and design of suspension systems.
	
	The state space representation was the mathematical model used in our model representation. It was used because it provides a concise and systematic way to represent the evolution of a system over time.
	
	The conventional two DOF quarter-car model was modified to include the inertia effects to simulate our real model and setup.
	
	After that, the result of dynamic response analysis and comparing between the simplified (conventional) and Inertial Quarter-car Suspension model. The analysis concluded that Sprung and Unsprung acceleration curves of the inertial model exceed those of the simplified model by 14.3\% and 3.8\%, respectively.
	
	A bump Profile is described as a road disturbance. These bumps simulate irregularities or variations in the road surface that can impact a vehicle's suspension system.
	
	The inputs to our system were different road input profiles as follows:
	\begin{itemize}
		\item Sinusoidal Bump
		\item Trapezoidal Bump
		\item Parabolic Bump
	\end{itemize}
	Then the results in terms of displacement, acceleration, suspension travel and tire deflection were demonstrated with different road input profiles and with including the tire radius effect.
	
	\item In the validation of simulation model section was comparing the data revealed from the real model with the simulation model data and was demonstrated the difference between them and the purposes that led to those differences
\end{itemize}

\section{Next Semester Work}
The study to this point not terminated but was very promising to go with improvement and enhancing such as:
\begin{itemize}
	\item Including each effect of the suspension elements in the simulation model.
	\item Including a parametric study and show the effect of changing the values of these suspension parameters.
\end{itemize}
Once this study and additional investigation is done, will let us to get into the upgrading to active suspension and the modification of our model to adapt the installation of the additional elements to enhance the system response.

\newpage

\fi