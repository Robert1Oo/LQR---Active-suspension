\chapter*{ACKNOWLEDGEMENTS}
First and foremost, we would like to express our deepest gratitude and heartfelt thanks to our families for their endless  support, understanding, and encouragement. Your belief in us, even during the most challenging moments and failures, has been our greatest source of motivation. This achievement would not have been possible without your love and sacrifices. To our friends and colleagues, thank you for your encouragement and shared experiences that have made this journey memorable and rewarding. We are also grateful to our college (Ain Shams University, Faculty of Engineering) for providing us with the resources and opportunity to pursue this project and achieve our academic goals. This milestone is a testament to the collective effort of everyone who stood by us, and we are forever thankful. And finally without our supervisors we would not do such an amazing work, thanks a lot.

\chapter*{ABSTRACT}
This thesis investigates the performance of various advanced control techniques applied to an active suspension system using a quarter-car model. The quarter-car model is widely used due to its simplicity and effectiveness in capturing the essential dynamics of a suspension system while allowing for efficient analysis and simulation. Simulation experiments are conducted to evaluate the effectiveness of these control algorithms.\\

The study focuses on the application of Linear Quadratic Regulator (LQR) for linear-time-invariant systems, Reinforcement Learning (RL), and Sliding Mode Control (SMC) to enhance suspension performance. LQR is particularly powerful for its ability to provide optimal control solutions by minimizing a defined cost function, balancing performance and energy efficiency. RL, on the other hand, offers a data-driven approach that adapts to system uncertainties and non-linearities, making it highly suitable for complex and dynamic environments. Additionally, the robustness of SMC makes it effective in handling system disturbances and parameter variations. Furthermore, the impact of state estimation using a Kalman filter is assessed, as it enables accurate estimation of unmeasured states, enhancing the overall control performance.\\

MATLAB/Simulink is employed to perform the simulations, beginning with the mathematical modeling of the system in state-space form, followed by the implementation of these modern control techniques. This research aims to contribute to a deeper understanding of control theory and its practical application in improving vehicle ride comfort and handling. By comparing the performance metrics of each control strategy, the study seeks to highlight the strengths and limitations of these advanced techniques, providing valuable insights into their applicability for real-world suspension systems.

\tableofcontents
\listoffigures
\listoftables

\chapter*{Nomenclature}

\begin{longtable}{lll}                                             \\
		
	\textbf{Symbol} & \textbf{Description} & \textbf{Unit} \\  
	$b_{s}$  & Damper average damping coefficient & N.s/m \\  
	$b_{tr}$ & Tire damping coefficient & N.s/m \\  
	$e$     & Signal error & - \\  
	$f$     & Frequency & Hz \\  
	$f_{n-s}$ & Sprung mass natural frequency & Hz \\  
	$f_{n-{us}}$ & Unsprung mass natural frequency & Hz \\  
	$k$     & Stiffness of the Suspension Spring & N/m \\  
	$k_{t}$ & Equivalent Spring Stiffness of the Tire & N/m \\  
	$L$     & Lenght of Conecting rod & mm \\  
	$M$     & Sprung Mass & kg \\  
	$m$     & Unsprung Mass & kg \\  
	$R$     & Lenght of Crank & mm \\  
	$S_D$   & Sprung displacment & mm \\  
	$STD$   & Static Tire Deflection & mm \\  
	$T$     & Periodic Time & s \\  
	$X$     & Position of the slider & mm \\  
	$z$     & Vertical Position of The Car Body & mm \\  
	$z_{{r}}$ & Road Excitation Displacement & m \\  
	$z_{{t}}$ & Vertical position of Unsprung Mass & m \\  
	$\theta$ & Angle of the crank & degrees \\  
	
	
	%$K_p$              & Propetional gain                                      & -             \\
	%$K_I$              & Intergration gain                             		   & -             \\
	%$K_d$              & Differential gain                 			            & -             \\

	%$N_p$              & Negative big                           				   & -             \\
	%$N_M$              & Negative medium                            			   & -             \\
	%$N_S$              & Negative small                            				  & -             \\
	%$P_S$              & Positve Small                             				 & -             \\
	%$P_M$              & Positve medium                           				   & -             \\
	%$P_B$              & Positve big                             				 & -             \\

	
	
\end{longtable}

\chapter*{Abbreviations}
\thispagestyle{plain}
\begin{flushleft}
	\begin{tabular}{ll}
		
	\textbf{Symbol} & \textbf{Description}\\	
	%AC   & Alternating Current \\
	ASS  & Active Suspension System \\
	%DC   & Direct Current \\
	DOF  & Degree-of-Freedom \\
	DTD  & Dynamic Tire Deflection \\
	EMI  & Electromagnetic Interference \\
	FA   & Actuator Force \\
	FLC  & Fuzzy Logic Control \\
	IMU  & Inertial Measurement Unit \\
	LQR  & Linear Quadratic Regulator \\
	LVDT & Linear Variable Displacement Transducer \\
	MPC  & Model Predictive Control \\
	MR   & Magnetorheological \\
	PI   & Performance Index \\
	PSS  & Passive Suspension System \\
	%PWA  & Piece-wise Affine Function \\
	RF   & Radio Frequency \\
	RL   & Reinforcement learning \\
	RMS  & Root Mean Square \\
	RPM  & Revolution per minute \\
	SD   & Sprung Mass Displacement \\
	SMC  & Sliding Mode Control \\
	ST   & Suspension Travel \\
	STD  & Static Tire Deflection \\
	%TOF  & Time-of-Flight \\
	TR   & Transmissibility Ratio \\
	UD   & Unsprung Mass Displacement \\
	%VFD  & Variable Frequency Drive \\
	%VSS  & Variable Structure Systems \\
	    %FDM  & Fused Deposition Modeling \\
		%NMPC & Nonlinear Model Predictive Control \\
		%eMPC & Explicit Model Predictive Control \\
		
		
		
	\end{tabular}
\end{flushleft}