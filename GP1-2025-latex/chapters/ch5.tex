\section{Conclusion}
This study demonstrated that the sliding mode controller (SMC) outperforms the linear quadratic regulator (LQR). As expected from the literature, the LQR provides an improvement over passive control but lacks the robustness and adaptability of advanced control strategies. While reinforcement learning (RL) has shown promise in recent studies, a direct comparison was not feasible due to the extensive training and tuning required. Overall, SMC proves to be a more effective solution for achieving superior performance and robustness in this application.


\section{Future Work}
In the next semester we  will implement, and experimentally validate a Hardware-in-the-Loop (HIL) results for the active suspension system on a quarter-car model test rig. This will involve:

\begin{itemize}
	\item Integrating the previously developed simulation models with the real quarter-car model test rig.
	\item Developing and implementing data acquisition and the simulated control algorithms within the HIL environment.
	\item Conducting rigorous experimental testing to compare the performance of the simulated and real-time implementations of the active suspension system under various road excitations.
	\item Analyzing the experimental results to assess the accuracy and effectiveness of the control algorithms, identify any difference between the simulated and real-world behavior of the quarter-car model, and quantify the achieved ride comfort and handling improvements.
	\item Investigating and mitigating the effects of sensor noise and actuator limitations on the system's performance within the context of the quarter-car model test rig.
	\item Refining the control algorithms based on the experimental findings to optimize the suspension system's ride comfort, handling, and road holding capabilities on the quarter-car model.
\end{itemize}